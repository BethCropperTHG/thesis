% Chapter 1

\chapter{Introduction} % Main chapter title}
\label{Introduction} % For referencing the chapter elsewhere, use \ref{Chapter1} 

\lhead{Introduction \emph{}} % This is for the header on each page - perhaps a shortened title

%----------------------------------------------------------------------------------------
\label{Introduction}


\section{Neutrinoless Double-Beta Decay}

The neutrino, according to the standard model of particle physics, is a massless, weakly interacting fermion\cite{manchesterparticle}. It has a lepton number of one. It comes in three flavours, one for each charged lepton, and has a corresponding antineutrino with an opposite lepton number and chirality.

However, this picture of the neutrino is not accurate. In the 1960s, the Sun was observed to emit a third of the neutrinos that were expected\cite{homestake, solarneutrinoproblem}. This was eventually realised to be because the neutrinos were oscillating between three flavours, one for each of the charged leptons. This implies that the neutrino is not massless, because oscillation implies the experience of time, which does not happen for massless particles.

The verification of neutrino oscillation by Super-Kamiokande\cite{superkamiokande} and the Sudbury Neutrino Observatory\cite{sudbury} at the turn of the millenium provided an experimentally observable break from the standard model of particle physics.

One potential implication of the neutrino possessing a mass is the potential for the neutrino to be a Majorana fermion, i.e. its own antiparticle\cite{majorana}. This would be observable through $0\nu2\beta$ decay. 

Double beta decay happens when a single beta decay is energetically forbidden, but a double beta decay is not. The isobar of the parent nucleus with one more proton has a lower binding energy, but the isobar with two protons more does not. In this case, two beta decays happening simultaneously is possible. However, the matrix elements for these decays are very small so the double beta decay lifetime is very large, of the order of $10^{17}$-$10^{24}$ years\cite{krane}, as per Fermi's Golden rule.

Normally, two neutrinos are emitted to balance the lepton number of the reaction, but if neutrinos are Majorana fermions, $0\nu2\beta$ decays are possible, which would be detectable. There would be a peak at the high energy limit of the double beta decay spectrum where the $\beta$ particles contain all of the energy released from the decay.

Experiments such as CUORE\cite{cuore}, EXO\cite{exo}, GERDA\cite{gerda}, KamLAND-Zen\cite{kamlandzen}, MAJORANA\cite{majoranaexpt}, and XMASS\cite{xmass} are looking for evidence of $0\nu2\beta$ decay, so far with no success. COBRA\cite{cobra} is an experiment particularly interesting with respect to this report, as the isotope to be studied is \textsuperscript{116}Cd. The reaction is the decay of \textsuperscript{116}Cd to \textsuperscript{116}Sn, and the nuclear structure of these is discussed in this report.

\section{The Relevance of Nuclear Structure}

This report discusses neutron transfer reactions on the $0\nu2\beta$ decay parent and daughter candidates \textsuperscript{116}Cd and \textsuperscript{116}Sn. The motivation for this is to aid theoretical predictions of the matrix element for the $0\nu2\beta$ decay between them. This will allow more accurate calculation of the mass of the neutrino.

The decay rate $\lambda$ for $0\nu2\beta$ decay is 

\begin{equation}
\lambda = G \, |M_{if}|^2 \, |m_{\beta\beta}|^2\mathrm{,}
\end{equation}

where $G$ is the phase space factor, $|M_{if}|$ is the matrix element, and $|m_{\beta\beta}|$ is the effective mass of the electron neutrino\cite{mbetabeta}.

If $0\nu2\beta$ decay is observed, $\lambda$ is also observed. $G$ can be calculated\cite{phasespace}. Therefore, if $|M_{if}|^2$ is known, $|m_{\beta\beta}|$ is calculable, which would yield the absolute mass scale of the neutrino, and help model neutrino oscillations\cite{mbetabeta}.

However, $|M_{if}|$ is not observable in any other reaction, so must be theoretically determined. $|M_{if}|$ represents the overlap between the initial and final states of the system, which contains the nuclear structure of both the parent and daughter nuclei. The occupancies of valence shell-model orbitals may change between the parent and daughter nuclei, which would mean that the nuclear structure changes between the parent and daughter nuclei. This would affect the matrix element.

The occupancies of these orbitals can be probed by single nucleon transfer. If it is difficult to add a single nucleon to an orbital, then the orbital must be nearly full, and vice-versa. Similarly, if it is easy to remove a nucleon from a particular orbital, it must also be nearly full, and vice-versa. This will be quantified later in the report.

\section{Report Summary}

This report discusses single-neutron transfer reactions on \textsuperscript{116}Cd, \textsuperscript{116}Sn, and \textsuperscript{114}Cd. This was done using a tandem accelerator in combination with a Q3D magnetic spectrometer. This was done to extract the occupancies of neutrons in valence single-particle orbitals.

These nuclei were chosen as \textsuperscript{116}Cd is a neutrinoless double-beta ($0\nu2\beta$) decay candidate, and \textsuperscript{116}Sn is the daughter nucleus of that reaction. \textsuperscript{114}Cd was chosen as a consistency check for the \textsuperscript{116}Cd.