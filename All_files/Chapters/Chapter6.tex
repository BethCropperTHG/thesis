% Chapter Template

\chapter{Conclusions} % Main chapter title

\label{Chapter6} % Change X to a consecutive number; for referencing this chapter elsewhere, use \ref{ChapterX}

\lhead{Chapter6. \emph{Conclusions}} % Change X to a consecutive number; this is for the header on each page - perhaps a shortened title

%----------------------------------------------------------------------------------------
%	SECTION 1
%----------------------------------------------------------------------------------------

\section{Summary}
(p,d) and (d,p) transfer reactions were done on \textsuperscript{116}Cd, \textsuperscript{114}Cd, and \textsuperscript{116}Sn. These were done using the Q3D magnetic spectrometer at the Maier-Leibnitz Laboratorium. These nuclei are interesting because \textsuperscript{116}Cd and \textsuperscript{116}Sn are the parent and daughter nuclei of a potential $0\nu2\beta$ reaction. The reactions were done to extract the occupancies of the valence orbitals, which helps when constructing models to determine the $0\nu2\beta$ matrix element. This allows the calculation of the neutrino mass upon the detection of the process.

The data from these reactions are currently being analysed. The yields have all been extracted. Angular distributions were extracted for the lowest-lying states in (p,d), and the angular distributions match what has already been measured. It just remains to do energy calibrations for the rest of the states, calculate spectroscopic factors from the peaks of the distributions, and sum them to find the occupancies. For the (d,p) reaction however, further review is required.
\newpage
%-----------------------------------
%	SUBSECTION 1
%-----------------------------------
\section{Outlook}

The spectroscopic factors and occupancies first need to be found for these targets. After this, (\textsuperscript{3}He, $\alpha$) data which was recently collected will be analysed, which better probes states of higher $\ell$ than proton or deuteron beams. As well as this, the (d,p) reaction has been repeated on the same targets for low excitation energies. This will be analysed to check that the cross-sections are consistent between that experiment and the experiment discussed in this report. All of these occupancies will also be compared to occupancies in the theoretical frameworks for the matrix elements for $0\nu2\beta$ decay to check that those frameworks reproduce the reality of the nuclear structure.

With the arrival of more experimental data comes the need for more sophisticated data analysis. Currently, the gf3 package from Radware is used to fit spectra. An automatic fitting program will be designed to replace it. This will need to be able to identify multiplet states and contaminants and extract yields for all states. Preliminary discussions suggest using machine learning to identify the nature of each peak will be beneficial.

